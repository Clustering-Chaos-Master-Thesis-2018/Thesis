\chapter{Introduction}
The technology of Wireless Sensor Networks (WSNs) is an active research area \cite{Yick2008-wsn-survey, Mahmood2015-reliability-survey}, it is a technique that enables small low-powered computer nodes to work in cooperation. The first application of Wireless Sensor Networks was for different military applications such as target tracking and troop movements \cite{Yick2008-wsn-survey}. However, WSNs have quickly expanded to many different fields, such as medical equipment, tracking natural disasters, and monitoring the weather. The nodes are battery powered, with limited processing and storage capabilities. Equipped with a low-power radio, they can communicate with other nearby nodes. They also use sensors to collect data which is then input into some computation and forwarded to a base station or disseminated throughout the network.

One advantage of WSNs is the ability to deploy nodes in the network in an ad hoc manner. Furthermore, the nodes' low powered nature make them inexpensive to manufacture. However, there are multiple challenges which require thorough consideration when implementing protocols for a WSN \cite{Yick2008-wsn-survey}. Since nodes are battery powered, they should restrict the time they spend sending, receiving and processing data to save energy. Another difficulty is maintaining high message propagation speed while not flooding the network with too many messages, hindering propagation of other messages. Additionally, scaling is a challenge; restrictions on the number of nodes in a WSN appear in many forms. For example, due to the maximum size of the transmitted packet, the message distribution algorithm, and node placement since a high node density can constrain message propagation.

The Chaos protocol \cite{chaos-introduction-paper}, presented in 2013, was the first protocol built for WSNs to have native support for all-to-all data sharing. Chaos builds on two core mechanisms. First, synchronous transmissions which means that the nodes in a network follow a global schedule that tells them when to wake up and either transmit or receive data. Second, user-defined merge operators which consist of some code that defines how the node processes and merges the data it receives. For example, calculating the maximum over a set of proposed values. Using these two mechanisms a node running the Chaos protocol can independently decide what action to perform in the next slot (sending data, receiving data, or doing nothing). Additionally, all processing (the execution of a merge operator) is part of the network protocol, just after a node transmits or receives data.

Further development of the Chaos protocol resulted in \atwo{} \cite{a2-introduction-paper}. \atwo{} addressed some shortcomings of the Chaos protocol and also introduced the \atwo{} \textit{Synchrotron}, a synchronous transmission kernel which has several features: frequency hopping, high precision time synchronisation, and the ability to schedule multiple applications to run in the network at different intervals. On top of Synchrotron, \atwo{} implements distributed consensus protocols such as two- and three-phase commit. Due to the new communication model in Chaos and the new features in the \atwo{} protocol, both Chaos and \atwo{} showed significant improvement in performance and reliability compared to similar protocols \cite{chaos-introduction-paper, a2-introduction-paper}.

A typical way to improve the network lifetime and reliability in WSNs is to use clustering \cite{Afsar2014-clustering-survey, Younis2006-clustering-survey}. Clustering is the practice of electing some set of nodes as \emph{Cluster Heads}(CH) and assigning each remaining node to one of these CHs. Nodes selecting a common CH belong to the same cluster and will primarily communicate with that subset of the network. The benefit of clustering is primarily in two different areas. First, since each CH only has to communicate with a subset of the network the amount of data the WSN needs to handle arguably becomes smaller by a factor approximately equal to the number of CHs in the network; however, there is added overhead for clustering the network and communication between CHs. Second, the CHs can aggregate and filter out redundant data from their clusters. This can be beneficial since a common task for a WSN is to forward all data it gathers to a base station. If only the CHs aggregate and forward the data to the base station, the network as a whole can save energy. 

There are many ways of applying clustering in a WSN; they differ primarily in the way the scheme decides the set of CHs. We provide a comprehensive categorisation of different clustering schemes in \cref{sec:background:clustering}.


\section{Limitations}
We impose several limits on the design and implementation of our clustering algorithm. We will not design a new clustering algorithm specifically for the \atwo{} protocol. We will consider an existing clustering algorithm and only make minor modifications to it to make the algorithm work on top of the \atwo{} protocol. Furthermore, we will only provide a basic reference implementation of the clustering algorithm using the \atwo{} Synchrotron as a base.

 \section{Purpose}
The purpose of this thesis is to investigate the effects of exchanging the current all-to-all protocol used in \atwo{} with a protocol based on clustering. The goal is to save energy and make the transfer of information more efficient when compared to the original \atwo{} protocol. We implement a clustering algorithm with modifications to fit \atwo{}s architecture. We also evaluate our implementation and compare it to \atwo{} to see if there is any difference concerning energy usage, latency, and reliability.

\section{Contributions}
Below, we present a brief summary of our main contributions.

\begin{itemize}
    \item We design a clustering scheme for the \atwo{} protocol, based on the HEED algorithm.
    \item We provide a reference implementation of the clustering scheme.
    \item We evaluate our reference implementation in a simulator and on the Flocklab testbed.
\end{itemize}

\section{Outline}
We organise this thesis as follows. In \cref{chap:background} we introduce relevant background about WSNs, the Chaos and \atwo{} protocols as well as clustering. In \cref{chap:related-research} we present research on clustering and communication protocols in WSNs. Following that, in \cref{chap:design} we present our design of the clustering implemented on top of \atwo{}. Next, in \cref{chap:implementation} we talk about the implementation specific details and issues encountered while implementing clustering on top of \atwo{} in Contiki. We present our evaluation of the clustering algorithm and comparison to the existing \atwo{} protocol in \cref{chap:evaluation}. Finally, we conclude in \cref{chap:conclusion}.
