\chapter{Introduction}
The technology of Wireless Sensor Networks (WSNs) is an active research area \cite{Yick2008-wsn-survey, Mahmood2015-reliability-survey}; it enables small low-powered computer nodes to work in cooperation. The first application of Wireless Sensor Networks was for different military applications such as target tracking and troop movements \cite{Yick2008-wsn-survey}. However, WSNs have quickly expanded to many other fields, such as natural disaster tracking and weather monitoring. The nodes are battery powered, with limited processing and storage capabilities. Equipped with a low-power radio, they can communicate with other nearby nodes. They also use sensors to collect data about the environment around them. This data is then used as input into some computation and either forwarded to a base station or disseminated throughout the network.

One advantage of WSNs is the ability to deploy nodes in the network in an ad hoc manner. Furthermore, the nodes' low powered nature make them inexpensive to manufacture. However, there are multiple challenges which require thorough consideration when implementing protocols for a WSN \cite{Yick2008-wsn-survey}. Since nodes are battery powered, they should restrict the time they spend sending, receiving, and processing data. Another difficulty is maintaining high message propagation speed while not congesting the network. Additionally, scaling is a challenge and restrictions on the number of nodes that can participate in a WSN appear in many forms. For example, the maximum size of the transmitted packets, the communication protocol, and the network topology, can all impact the scalability of a WSN.


The Chaos protocol \cite{chaos-introduction-paper}, presented in 2013, is the first protocol built for WSNs to have native support for all-to-all data sharing. Chaos builds on two core mechanisms: synchronous transmissions and user-defined merge operators. Synchronous transmissions mean that the nodes in a network follow a global schedule that tells them when to wake up and either transmit or receive data. User-defined merge operators consist of some code that defines how a node processes and merges received data. For example, calculating the maximum value over a set of proposed values. By using these two mechanisms, a node running the Chaos protocol can independently decide what action to perform in the next slot: sending data or receiving data. Additionally, the nodes perform all processing (the execution of a merge operator) as part of the network protocol after they transmit or receive data.

Further development of the Chaos protocol resulted in \atwo{} \cite{a2-introduction-paper}. \atwo{} addresses some shortcomings of the Chaos protocol and also introduces the \atwo{} \textit{Synchrotron}, a synchronous transmission kernel which has several features: frequency hopping, high precision time synchronisation, and the ability to schedule multiple applications to run in the network at different intervals. On top of Synchrotron, \atwo{} implements distributed consensus protocols such as two- and three-phase commit. Due to the new communication model in Chaos and the new features in the \atwo{} system, both Chaos and \atwo{} shows significant improvement in performance and reliability compared to similar protocols \cite{chaos-introduction-paper, a2-introduction-paper}.

A typical way to improve the network lifetime and scalability in WSNs is to use clustering \cite{Afsar2014-clustering-survey, Younis2006-clustering-survey}. Clustering is the practice of electing a set of nodes as \emph{Cluster Heads} (CH) and assigning each remaining node to one of these CHs. Nodes selecting the same CH belong to the same cluster and will only communicate within that subset of the network. The benefit of clustering is primarily reducing the number of packet transmissions and consequently the amount of data the network has to handle. Since each CH only has to communicate with a subset of the network, the number of packet transmissions in each cluster is reduced by a factor approximately equal to the number of CHs in the network. Furthermore, the CHs can aggregate and filter out redundant data from within their clusters, before forwarding it to a base station, or disseminating the aggregate to the network.


\begin{newtext}{}
\section{Problem and Aim}
Building on \atwo{} Synchrotron, we aim to increase network lifetime and improve scalability while maintaining \atwo{}'s high probability of reaching consensus, which is higher than 99\%. To achieve these goals, we extend \atwo{}'s design with a clustering mechanism. We evaluate the implementation of the new design and compare it to the \atwo{} Synchrotron using the following metrics:

\begin{itemize}
    \item Reliability measures the success rate of an application running on the network.
    \item Stability measures the number of topology changes that occur for the network.
    \item Latency measures how quick a network executes an application and terminates.
    \item Energy usage measures the energy consumed by the radio and CPU of a node.
\end{itemize}

\end{newtext}

\section{Limitations}
We impose several limitations on the design and implementation of our clustering algorithm. We will not design a new clustering algorithm specifically for the \atwo{} system. We will consider an existing clustering algorithm and only make minor modifications to implement the algorithm on the \atwo{} system. Furthermore, we will only provide a reference implementation for the Clustering service, and not implement features such as fault tolerance for crashing nodes or corrupted data.


\section{Contributions}
In this thesis, we make the following contributions.

\begin{itemize}
    \item We design a clustering scheme for the \atwo{} system, based on the HEED algorithm \cite{Younis2004-HEED}.
    \item We provide an implementation of the clustering scheme in Contiki OS \cite{Dunkels2004-contiki-introduction}.
    \item We evaluate our reference implementation in Cooja \cite{Osterlind2006-cooja-introduction}, the simulator for Contiki OS, and on the Flocklab testbed \cite{Lim2013-flocklab-introduction}.
\end{itemize}

\section{Thesis Outline}
We organise the rest of this thesis as follows. In \cref{chap:background} we introduce WSNs, Chaos and the \atwo{} system, as well as the areas of clustering and gossiping. In \cref{chap:related-research} we present research on clustering and another thesis attempting to increase the scalability of Chaos. In \cref{chap:design} we present our design of the clustering implemented on top of \atwo{}. In \cref{chap:implementation} we talk about the implementation specific details and issues encountered while implementing clustering on top of \atwo{} in Contiki. In \cref{chap:evaluation}, we present our evaluation of the clustering algorithm and comparison to the existing \atwo{} system. Finally, in \cref{chap:conclusion}, we conclude and list future work.
