\chapter{Conclusion}
\label{chap:conclusion}

Improving the scalability and energy usage in protocols for Wireless Sensor Networks is a research question with much focus. A common way to do this is to cluster the network. By partitioning the network into separate clusters, the network spends less energy on communication and can scale to more nodes, because each cluster only has to handle a subset of the network. In this thesis, we introduced clustering to the \atwo{} protocol, an all to all communication protocol which supports distributed consensus in low powered wireless networks. The clustering implementation is based on the HEED algorithm, and it uses the residual energy of the nodes to weigh their probabilities to announce themselves as cluster heads. We implemented clustering as a service in the \atwo{} Synchrotron. Additionally, we implemented a demotion service, which removes clusters that are considered too small.

We evaluated our implementation in the Cooja simulator and on the Flocklab testbed looking at the metrics, reliability, latency, and energy consumption. The application we used in our evaluations is the max aggregate, it determines the maximum value proposed by a node in the network, and disseminates that value to all nodes. The results from our evaluation show that our clustering implementation performs significantly worse than \atwo{} when comparing average reliability; also the variance is high in the cluster implementation compared to \atwo{}. However, when increasing the nodes from 50 to 200, our clustering implementation achieves both lower latency and a slightly better energy consumption than the original \atwo{} protocol. However, further work is required to determine the cause of the high variance in the reliability results, and until this is fixed, we cannot draw any definite conclusions about the performance of our clustering implementation.

\section{Future Work}
\label{sec:future-work}
In this section, we list and discuss possible focuses of future work.


\subsection{Sleep Schemes}
Since forwarders do not contribute to the completion of the application, i.e., they do not set contribution flags. Sleep schemes apply to our clustering implementation since we showed in \cref{chap:evaluation} that using clustering does not increase the energy usage of the node in the network. Therefore, applying sleep schemes will directly improve energy preservation proportionally to the percentage of nodes put to sleep; this, however, could be in exchange for reliability. Sleep schemes are very applicable when using clustering, and further work can research which nodes can be put to sleep, and how to determine which are crucial for keeping a network connected.

\subsection{Transmission Power}
In this thesis, all nodes have always used the same transmission power. It is conceivable to use a stronger transmission power for CHs that scales with competition radius, enabling all non-CHs to sleep during CH rounds. If the few elected CHs use a higher transmission power, all other nodes, currently acting as forwarders, can sleep and save energy. However, it would put a high requirement on dynamic clustering since when normal nodes sleep, CHs will continue to use energy. Dynamic clustering would enable CHs to schedule the clustering service which should not elect them again.


\subsection{Evaluating Stability}
In our evaluation, we redefined the reliability metric slightly compared to Landsiedel et al.~\cite{chaos-introduction-paper}. We only consider reliability for the max application and count other applications as an automatic failure, if they are run when not expected. In future evaluations, we could define reliability like Landsiedel et al.~does, the number of rounds in which the network reaches completion regardless of which application is running. Furthermore, we could also introduce a new metric called stability.

Stability would measure how often a node requires resynchronisation. The measurement would describe, in the case of \atwo{}, how connected the network is; and in the case of clustering, how many times nodes switch between clusters. Comparing the stability metric between these two could provide a better picture of the overall effectiveness of clustering a network. While stability would measure how likely the clustering algorithm is to produce good clusters, reliability would indicate how well an application runs on top of a clustered network.

\subsection{Using Other Clustering Algorithms}
In our work, we based our clustering implementation on the HEED algorithm. Its primary goal is to save energy, which is a common goal for clustering algorithms. However, further work could investigate if better clustering could be achieved by using another clustering algorithm as a base. For example, an algorithm that is more deterministic, since our evaluation showed that our clustering implementation has high variance, especially when measuring reliability.
