\chapter{Conclusion}
\label{chap:conclusion}

Improving scalability and decreasing energy usage in protocols for Wireless Sensor Networks is a research question with much focus. A typical way to accomplish this is to cluster the network. By partitioning the network into separate clusters, the network spends less energy on communication and can, therefore, scale to more nodes. In this thesis, we have introduced clustering to \atwo{}, a system that enables distributed consensus in low powered wireless networks. The clustering implementation is based on the HEED algorithm, which maximises network lifetime by taking into account the residual energy of nodes when electing cluster heads. We have implemented a clustering scheme in the \atwo{} Synchrotron.

We have evaluated our implementation in the Cooja simulator and on the Flocklab testbed using the metrics stability, reliability, latency, and energy consumption. The results showed that our clustering implementation achieved similar reliability but with higher variance in stability. However, for the larger networks we evaluated, with 200 nodes, our clustering implementation achieved both lower latency and better energy consumption than the \atwo{} system. We determined that the variance in stability for our clustering implementation was due to our limitation regarding fault tolerance. Future work should, among other things, investigate and implement fault tolerance mechanisms.

\section{Future Work}
\label{sec:future-work}
In this section, we list and discuss possible goals for future work. With a focus on implementing fault tolerance measures, which we specified as a limitation, and further evaluation to reveal how clustering affects the \atwo{} system more thoroughly.


\subsection{Sleep Schemes}
Sleep schemes control which nodes wake up each round and which nodes do not. Since forwarders only route packets during CH rounds, sleep schemes are applicable for those nodes. Furthermore, since we showed in \cref{chap:evaluation} that a  clustered network requires less energy than a normal network, sleeping nodes would further decrease the energy consumption of the network proportional to the percentage of nodes put to sleep; this, however, could be in exchange for reliability. Further work would research ways of determining which nodes can be put to sleep without affecting the stability or reliability of the network.

\subsection{Transmission Power}
In this thesis, all nodes have used the same transmission power. It is conceivable to use a stronger transmission power for CHs that increases with a greater competition radius, enabling all other nodes to sleep during CH rounds. However, it would put a high requirement on dynamic clustering since when normal nodes sleep, CHs will continue to use energy. Dynamic clustering would enable CHs to schedule the Clustering service which should elect new CHs and evenly distribute the energy required by the network.

\subsection{Fault Tolerance}
\begin{newtext}
As we discussed in \cref{subsec:discussion-stability-and-fault-tolerance} and specified in our limitation, we did not consider fault tolerance for our implementation. As such, there are several faulty scenarios which our implementation cannot handle. The most important one is the scheduling of the Join service. To properly schedule the Join service locally in each cluster, without affecting the other nodes during CH rounds, would increase stability significantly.
\end{newtext}

\subsection{Measuring Throughput}
\begin{newtext}
In addition to the metrics we used in this thesis, it would be interesting to measure the throughput the network can handle, and if there is any difference when clustering is applied. Furthermore, performing these evaluations would give us data on how the clustering performs when the packet size is increased. The max application only used a small amount of the maximum packet size, which could affect the results.
\end{newtext}

\subsection{Using Other Clustering Algorithms}
In our work, we based our clustering implementation on the HEED algorithm.  However, we have not analysed the effectiveness of the algorithm itself, and it is possible that another algorithm would produce better results when applied to the \atwo{} system. Further work could investigate what happens using a different clustering algorithm, with a focus on different classes of algorithms, such as deterministic or centralised algorithms.